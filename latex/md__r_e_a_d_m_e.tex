An artificial convolutional neural network for the visible wavelength region of stellar spectra

For detailed information about this small, simple library, and an easy way to use the neural network, please visit $<$website here$>$

Quick Notes\+:


\begin{DoxyItemize}
\item If creating your own neural networks using \mbox{\hyperlink{namespace_stella_net}{Stella\+Net}} as a starting point, it is recommended to run the effective temperature, log g, and \mbox{[}M/H\mbox{]} branches of the neural network trainer separately, since they converge at different rates. The neural network included in the package works well for visible spectra with the grid I generated (detailed description on the \mbox{\hyperlink{namespace_stella_net}{Stella\+Net}} website, some brief details below).
\item If your data set is not perturbed enough (i.\+e. the synthetic spectra you use do not cover a wide variety of S\+NR, vsini, metallicity, individual abundances, etc.) then your neural network will be very quickly overfit. The default models have been generated as follows\+:
\item Effective temperatures\+: \mbox{[}3500K -\/ 13000K\mbox{]} with step size 250K
\item log g values\+: \mbox{[}2.\+0 -\/ 5.\+0\mbox{]} with step size 0.\+5
\item \mbox{[}M/H\mbox{]} values\+: \mbox{[}-\/3.\+00 -\/ 0.\+5\mbox{]} with step size 0.\+25
\item S\+NR values\+: 150, 300 (random gaussian noise)
\item vsini values\+: 5, 50, 100, 200, 300
\item Atomic abundances up to Fe were generated for each individual model as being a gaussian distribution around the \mbox{[}M/H\mbox{]} value with minima and maxima +-\/ 1 from \href{to prevent overfitting of [M/H]}{\texttt{ M/H}}
\end{DoxyItemize}

This kind of distribution is recommended at a minimum to get a network that converges well for each parameter.

All synthetic spectra were generated with Kurucz/\+Castelli\textquotesingle{}s model atmospheres (\href{http://wwwuser.oats.inaf.it/castelli/grids.html}{\texttt{ http\+://wwwuser.\+oats.\+inaf.\+it/castelli/grids.\+html}}) and the stellar spectral synthesis code used was S\+P\+E\+C\+T\+R\+UM 2.\+76e by Richard Gray (\href{http://www.appstate.edu/~grayro/spectrum/spectrum.html}{\texttt{ http\+://www.\+appstate.\+edu/$\sim$grayro/spectrum/spectrum.\+html}})

i\+Spec by Sergi Blanco-\/\+Cuaresma was modified to facilitate simple generation of the synthetic spectrum grid. I also referenced his code for the vsini perturbation function in \mbox{\hyperlink{namespace_stella_net}{Stella\+Net}} (which was itself derived from some older routines). I highly recommend you check out his work at \href{https://www.blancocuaresma.com/s/iSpec}{\texttt{ https\+://www.\+blancocuaresma.\+com/s/i\+Spec}}

February 10, 2020 